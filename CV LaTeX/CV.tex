\documentclass[11pt,a4paper,sans]{moderncv}        % possible options include font size ('10pt', '11pt' and '12pt'), paper size ('a4paper', 'letterpaper', 'a5paper', 'legalpaper', 'executivepaper' and 'landscape') and font family ('sans' and 'roman')

% modern themes
\moderncvstyle{banking}                            % style options are 'casual' (default), 'classic', 'oldstyle' and 'banking'
\moderncvcolor{blue}                                % color options 'blue' (default), 'orange', 'green', 'red', 'purple', 'grey' and 'black'
%\renewcommand{\familydefault}{\sfdefault}         % to set the default font; use '\sfdefault' for the default sans serif font, '\rmdefault' for the default roman one, or any tex font name
%\nopagenumbers{}                                  % uncomment to suppress automatic page numbering for CVs longer than one page

% character encoding
\usepackage[utf8]{inputenc}                       % if you are not using xelatex ou lualatex, replace by the encoding you are using
%\usepackage{CJKutf8}                              % if you need to use CJK to typeset your resume in Chinese, Japanese or Korean

% adjust the page margins
\usepackage[scale=0.75]{geometry}
%\setlength{\hintscolumnwidth}{3cm}                % if you want to change the width of the column with the dates
%\setlength{\makecvtitlenamewidth}{10cm}           % for the 'classic' style, if you want to force the width allocated to your name and avoid line breaks. be careful though, the length is normally calculated to avoid any overlap with your personal info; use this at your own typographical risks...

\usepackage{import}

\moderncvcolor{blue}


% personal data
\name{Jousef}{Murad	}
\title{Curriculum Vitae}                               % optional, remove / comment the line if not wanted
\address{Waldst\"uckerring 50, 76756 Bellheim - Germany}{}{}% optional, remove / comment the line if not wanted; the "postcode city" and and "country" arguments can be omitted or provided empty
\phone[mobile]{+49 1577 9298171}                   % optional, remove / comment the line if not wanted
\phone[fixed]{07272 76834}                    % optional, remove / comment the line if not wanted
%\phone[fax]{+3~(456)~789~012}                      % optional, remove / comment the line if not wanted
\email{jousef.m@googlemail.com}                               % optional, remove / comment the line if not wanted
\homepage{www.engineered-mind.com}                         % optional, remove / comment the line if not wanted
%\extrainfo{TEST information}                 % optional, remove / comment the line if not wanted
\social[linkedin][www.linkedin.com/in/jousefmurad/]{Jousef-Murad}
\photo[64pt][0.4pt]{JOUSEF.png}                       % optional, remove / comment the line if not wanted; '64pt' is the height the picture must be resized to, 0.4pt is the thickness of the frame around it (put it to 0pt for no frame) and 'picture' is the name of the picture file
\quote{"I could either watch it happen or be a part of it." \\ - Elon Musk}                                 % optional, remove / comment the line if not wanted

% to show numerical labels in the bibliography (default is to show no labels); only useful if you make citations in your resume
%\makeatletter
%\renewcommand*{\bibliographyitemlabel}{\@biblabel{\arabic{enumiv}}}
%\makeatother
%\renewcommand*{\bibliographyitemlabel}{[\arabic{enumiv}]}% CONSIDER REPLACING THE ABOVE BY THIS

% bibliography with mutiple entries
%\usepackage{multibib}
%\newcites{book,misc}{{Books},{Others}}
%----------------------------------------------------------------------------------
%            content
%----------------------------------------------------------------------------------
\begin{document}
%\begin{CJK*}{UTF8}{gbsn}                          % to typeset your resume in Chinese using CJK
%-----       resume       ---------------------------------------------------------
\makecvtitle


\small{I am a mechanical engineer in the final year of a master's degree with focus on fluid mechanics. Passionate about science, with focus on turbulence modelling, developing interpersonal skills and interest in machine learning \& artificial intelligence as well as Entrepreneurship.}

\section{Foundations}

\vspace{6pt}

\begin{itemize}
	
\item{\cventry{June 2019}{Founder}{Engineered-Mind}{Bellheim}{}{\vspace{3pt} Website and YouTube channel (Jousef Murad) for engineering, programming, AI as well as psychology and self-development. \textbf{Goal: Inspire 1 million people!}}}

\end{itemize}

\section{Previous Employment}

\vspace{6pt}

\begin{itemize}
	
\item{\cventry{June 2019 - Present}{Tutor}{Karlsruhe Institute of Technology - Institute for Engineering Mechanics (ITM)}{Karlsruhe}{}{\vspace{3pt} Tutor students in the fields of numerical solutions of ordinary differential equations, numerical integration of differential-algebraic system of equations (DAEs), system with distributed parameters - fluid simulation with help of the finite difference method (FDM), numerical solution procedure for partial differential equations (PDE): finite element method (FEM) as well as verification \& validation}}


\item{\cventry{November 2018 - July 2019}{Research Assistant}{Karlsruhe Institute of Technology - Institute for Prod. Dev. (IPEK)}{Karlsruhe}{}{\vspace{3pt} Working on the implementation of cloud-based simulation technology (SimScale) into workshops of mechanical construction as well as doing intro lectures in front of 300+ people to show how to use simulation tools like SimScale.}}

\vspace{6pt}

\item{\cventry{October 2018 - December 2019}{Tutor}{Karlsruhe Institute of Technology - Institute for Mechanics (ITM)}{Karlsruhe}{}{\vspace{3pt}Responsible for more than 50 students and teach them about subjects like Computational Fluid Dynamics (CFD) and Finite Element Analysis (FEA) in the the field of \textbf{Modeling and Simulation} using MATLAB.}}

\vspace{6pt}

\item{\cventry{June 2018 - today}{Community \& Academic Program Manager}{SimScale GmbH}{Munich}{}{\vspace{3pt} Responsible for interactions inside the forum and providing full-time support to users including problem solving for simulations in the field of FEA, CFD and Thermal Analysis. Responsible for several Formula student teams all over the world making sure they get the best support possible and providing them with knowledge if needed. On top of that I am building the community by recruiting so called Power Users inside the SimScale forum to grow our presence and make sure to keep the forum vivid.}}

\vspace{6pt}

\item{\cventry{May 2018 - August 2018}{Research Assistant}{Karlsruhe Institute of Technology - Institute for Process Engineering}{Karlsruhe}{}{\vspace{3pt Investigating the physics of the Taylor-GreenVortex with the Lattice Boltzmann Method with strong focus on spectral methods and validation of the Kolmogorov spectrum using the tool FFTW.}}}

\vspace{6pt}


\item{\cventry{October 2017 - today}{Research Assistant}{Karlsruhe Institute of Technology - Institute for Fluid Mechanics (ISTM)}{Karlsruhe}{}{\vspace{3pt}Tutor for experimental fluid mechanics. Responsible for several groups of students teaching about density based measurement techniques and experiments including \textbf{Mach-Zehnder Interferometry} as well as \textbf{Schlieren Technique}.}}


\vspace{6pt}


\item{\cventry{July 2017 - June 2018}{Community Manager}{SimScale GmbH}{Munich}{}{\vspace{3pt} Engaging users in the forum helping them with getting their simulation done in the fields of CFD, FEA, Thermal Analysis.}}

\vspace{6pt}


\item{\cventry{July 2017 - August 2017}{Research Assistant}{Karlsruhe Institute of Technology - Institute for Prod. Dev. (IPEK)}{Karlsruhe}{}{\vspace{3pt} Working on the script for the lecture Product Development - Development Method.}}

\vspace{6pt}


\item{\cventry{November 2016 - July 2017}{FEA Simulation Assistant}{SimScale GmbH}{Munich}{}{\vspace{3pt} Setting up simulations in the field of FEA \& CFD. Providing full-time user support in and outside the forum as well as creating content for community building.}}

\vspace{6pt}


\item{\cventry{August 2016}{Tutor}{Studytutors}{}{}{\vspace{3pt} Doing webinars about mechanical design for 22 people.}}

\vspace{6pt}


\item{\cventry{June 2016 - August 2016}{Tutor}{Karlsruhe Institute of Technology - Institute for Prod. Dev. (IPEK)}{Karlsruhe}{}{\vspace{3pt} Giving tutorials about mechanical design for groups of five to six people.}}

\vspace{6pt}


\item{\cventry{February 2016 - June 2016}{Research Assistant}{Karlsruhe Institute of Technology - Institute of Fluid Machinery (FSM)}{Karlsruhe}{}{\vspace{3pt} Fluid solver code debugging with Alinea DDT.}}

\vspace{6pt}


\item{\cventry{May 2015 - September 2015}{Working Student}{APL GmbH}{Landau}{}{\vspace{3pt} Generation of real surfaces and evaluation of surface properties as well as investigation of surface parameters with the Fast-Fourier-Transform with MATLAB.}}

\vspace{6pt}


\item{\cventry{April 2015 - May 2015}{Working Student}{APL GmbH}{Landau}{}{\vspace{3pt} Data evaluation of tribological data as well as preparation and filtering of surfaces with MATLAB. CAD modelling of a V8 engine, dynamic animation for company presentations as well as rendering with Creo 3.0 \& Keyshot.}}

\vspace{6pt}


\item{\cventry{October 2014 - January 2015}{Internship}{APL GmbH}{Landau}{}{\vspace{3pt} Calculation of engine components with the Finite-Element-Method. Working in the field of tribology and contact mechanics.}}

\vspace{6pt}


\item{\cventry{February 2014 - October 2014}{Working Student}{Karlsruhe Institute of Technology - Institute for Mechanics (ITM)}{Karlsruhe}{}{\vspace{3pt} Investigation of material parameters for strain hardening with a dynamic-mechnical analyser with tensile tests for metal sheets and polymers.}}

\vspace{6pt}


\item{\cventry{September 2013 - October 2013}{Internship}{APL GmbH}{Landau}{}{\vspace{3pt} Basic internship in the field of basic machining methods, cutting methods, connection technology and CAD modelling with Pro-E for a self-built stirling engine.}}

\vspace{6pt}


\item{\cventry{February 2013 - September 2013}{Research Assistant}{Karlsruhe Institute of Technology - Institute for Prod. Dev. (IPEK)}{Karlsruhe}{}{\vspace{3pt} Contact simulations with Abaqus 6.12.}}

\vspace{6pt}


\item{\cventry{August 2011 - September 2014}{Working Student}{Daimler AG}{W\"orth}{}{\vspace{3pt} Working at the assembly lines for the Actros and Zetros trucks of Daimler.}}

\vspace{6pt}


\item{\cventry{2007 \& 2008}{Internship}{Schuler SMG GmbH \& Co. KG}{Wag\"ausel}{}{\vspace{3pt} Learning basic knowledge of computer applications and assembling as well as disassembling of computers. Using Microsoft Office with focus on Excel and its capabilities.}}

\end{itemize}

%\newpage
\vspace{5pt}

\subsection{Notable Projects}

\vspace{5pt}

\begin{itemize}
	
\item{\textbf{Own Project (starting soon):} \textit{'Application of Reinforcement Learning on Gaming'}
		
\vspace{3pt}
		
\small{In this self-taught project I will apply reinforcement techniques to play a ping-pong game and maybe to apply it to other games. Other potential games: Counter-Strike 1.6, Counter Strike Source or any other Ego-Shooter. Alternatively I want to try if I am able to teach a mid hero from DotA 2 how to contest the midlane.}}  \newline
			
\faGithub : Link will follow! \newline

\item{\textbf{Masters Project:} \textit{'Linear Stability Analysis for Plane Poiseuille flow'}

\vspace{3pt}

\small{Many problems in fluid mechanics involve some aspect of flow stability, analogous to solid mechanics. The basic question is: given a basic flow state (e.g. laminar flow through a pipe) under which conditions does the flow become unstable to certain perturbations? As a first step to determine the stability of a fluid flow problem, one often supposes that the perturbations to the basic state are of very small amplitude, which allows for a linearisation of the equations. Although this is a strong assumption, linear stability analysis has proven useful in many flow configurations.  \newline

In this assignment the stability of plane channel flow has been analyzed. The base flow was supposed to be fully-developed, pressure-driven, laminar flow directed in the x-direction (the y and z-directions are the wall- normal and spanwise coordinates, repectively). The distance between the plates is 2h. Only two-dimensional flow perturbations in the (x,y)-plane have been considered.}} \newline

\faGithub : \url{https://github.com/jousefm/Linear-Stability-Analysis-Poiseuille} \newline

\item{\textbf{Masters Project:}} \textit{'Lattice Boltzmann Method'}

\vspace{3pt}

\small{This report involved simulations for a Lid-Driven Cavity and the Kármán Vortex Street. Different code adaptions had to be made and several test cases have been carried out. } \newline

\faGithub : \url{https://github.com/jousefm/LBM-1}

\end{itemize}

\section{Education}

\vspace{6pt}

\subsection{Academic Qualifications}

\vspace{5pt}

\begin{itemize}

\item{\cventry{April 2017 - today}{Master Mechanical Engineering }{Karlsruhe Institute of Technology}{Karlsruhe}{}{}}

\item{\cventry{September 2011 - April 2017}{Bachelor Mechanical Engineering }{Karlsruhe Institute of Technology}{Karlsruhe}{}{}}

\item{\cventry{2007 - 2011}{High School}{Eduard-Spranger Gymnasium}{Landau}{}{}}  % arguments 3 to 6 can be left empty

\item{\cventry{2002 - 2007}{Primary School}{Grundschule Bellheim}{Bellheim}{}{}}  % arguments 3 to 6 can be left empty

\end{itemize}

\vspace{2pt}

%\newpage


\section{Thesis}
\vspace{3pt}

\subsection{Bachelor of Science with focus on "Construction and Validation of Mechanical Constructions"}

\vspace{3pt}

\textbf{Title:} Investigation of the modelling of real, technical surfaces \newline

\textbf{Supervisor:} Dipl.-Ing. Stefan Reichert \newline

\textbf{Description:} The thesis dealt with the analysis of statistical roughness parameters of numerical generated surfaces. For this purpose, the finite element software Abaqus is used with a plugin, which makes it possible to import topographies generated with a Matlab script allowing a contact simulation between two surfaces. All the relevant parameters have been evaluated in a post-processing step. An automatic report generator has been written showing the change of the so called Abbott-Firestone curve.

\vspace{6pt}

\subsection{Master of Science with focus on "Fluid Mechanics \& Computational Mechanics" \qquad \qquad}

\vspace{3pt}

\textbf{Title:} Predictive Maintenance and Explainable AI (currently working on, no fixed title yet) \newline

\textbf{Supervisor:} Nadia Burkart \newline

\textbf{Description:} Data driven prognostic systems enable us to send out an early warning of machine failure in order to reduce the cost of failures and maintenance and to improve the management of the maintenance schedule. For this purpose, robust prognostic algorithms such as deep neural networks are used whose put is often difficult to interpret and comprehend. We investigate these models with the aim of moving towards a transparent and understandable model which can be applied on critical applications such as within the manufacturing industry.

\vspace{6pt}

\section{Courses}

\begin{itemize}

\item{\cventry{currently working on}{Deployment of ML Models}{Udemy}{Online}{}{\vspace{3pt} This 6 part course teaches how to deploy a machine learning model and which tools to use}}
	
\item{\cventry{currently working on}{DeepMind Course}{DeepMind}{Online}{}{\vspace{3pt} This 18 part course teaches about Deep Learning and Reinforcement Learning}}
	
\vspace{6pt}
	
\item{\cventry{currently working on}{Introduction to Tensorflow for AI, ML and DL}{Coursera}{Online}{}{\vspace{3pt} This course from Andrew Ng and Laurence Moroney helps to build scalable AI-powered algorithms using Tensorflow}}
	
\vspace{6pt}
	
	
\item{\cventry{Finished with a certificate}{AI for Everyone}{Coursera}{Online}{}{\vspace{3pt} Learned about AI terminologies, state-of-the-art learning methods, how to implement AI into a company also taking into account technical, business and ethical diligence}}
	
\vspace{6pt}

\item{\cventry{Finished with a \href{https://www.udemy.com/certificate/UC-1CC5NC9O/}{certificate}}{Machine Learning A-Z: Hands-On Python \& R In Data Science}{Udemy}{Online}{}{\vspace{3pt} Creating Machine Learning Algorithms in Python (R was neglected)}}

\vspace{6pt}


\item{\cventry{(soon) September,12th - September, 14th 2018}{Short Course}{TU Dresden}{Karlsruhe}{}{\vspace{3pt} Numerical Calculation of turbulent flows in science and practice}}

\vspace{6pt}


\item{\cventry{March,19th - March, 23rd 2018}{Spring School}{Karlsruhe Institute of Technology}{Karlsruhe}{}{\vspace{3pt} Lattice Boltzmann Methods with OpenLB Software Lab}}

\vspace{6pt}


\item{\cventry{October 2017}{Training Course}{Karlsruhe Institute of Technology}{Karlsruhe}{}{\vspace{3pt} Introduction to the computational fluid dynamics with OpenFOAM. Learn the use of existent solvers and utilities as well as the extension and modification of solvers for own simulation purposes.}}

\vspace{6pt}


\item{\cventry{Summer Semester 2016}{CAE Workshop}{Karlsruhe Institute of Technology}{Karlsruhe}{}{\vspace{3pt} Learning about the Finite Element Method, topology optimization and shape optimization using the commercial software package Abaqus.}}
\end{itemize}



\vspace{6pt}

%\newpage

\section{Technical skills}

\vspace{6pt}

\subsection{Computer Languages}

\vspace{3pt}

\begin{itemize}

\setlength\itemsep{1em}

\item \textbf{Basic:} Python, Git

\item \textbf{Intermediate:} Maple

\item \textbf{Advanced:} Matlab, \LaTeX

\end{itemize}

\vspace{6pt}

\subsection{Modelling \& Simulation Software}

\vspace{3pt}

\begin{itemize}

\setlength\itemsep{1em}

\item \textbf{Basic:} IcemCFD, OpenFOAM

\item \textbf{Intermediate:} Ansys, Abaqus 6.12 - 6.14, Pro Engineer, Catia, Creo 2.0 \& 3.0, MS Office, Tecplot

\item \textbf{Advanced:} Paraview

\end{itemize}

\vspace{6pt}

\subsection{Operating Systems}

\vspace{3pt}

\begin{itemize}

\setlength\itemsep{1em}

\item \textbf{Basic:} -

\item \textbf{Intermediate:} Linux

\item \textbf{Advanced:} Unix/MacOS, Windows

\end{itemize}


\section{Personal skills}

\vspace{6pt}

\small{As a student employee being in several positions I was able to gather a lot of technial experiences as well as interpersonal skills/social competences and therefore improving my soft skills. \newline

My high degree of motivation in team works has always been appreciated by my team members. They describe me as creative, resourceful, inquisitive as well as goal oriented.}

%\newpage

\section{Scholarship}

\begin{itemize}
	
	\setlength\itemsep{1em}
	
	\item \textbf{Louis Schuler Fonds:}  December 2014 - July 2016
	
	\item \textbf{Louis Schuler Fonds:}  April 2018 - September 2019
	
\end{itemize}

\section{Ambassadorship}

\begin{itemize}

\setlength\itemsep{1em}

\item \textbf{GitKraken:}  July 2019 - today

\end{itemize}

\section{Languages}

\begin{itemize}

\setlength\itemsep{1em}

\item \textbf{German:} Mother tongue

\item \textbf{English:}  Advanced (C1 with certificate)

\item \textbf{French:}  Basic

\item \textbf{Arabic:}  Basic


\end{itemize}

\section{Interests}

\vspace{6pt}

\begin{itemize}

\item{I am very passionate about gaming and am playing DotA as well as the newer part DotA 2 for more than 8 years now. Related to that I am following the progress of OpenAI and their bot competing against the best player in the world by using reinforcement learning techniques} \newline

\item{I am currently learning the basics of the Chinese language} \newline

\item{Video editing is also one of my passions, although I am not very proficient in it I am always trying to learn something new to improve the quality of my YouTube channel}

\end{itemize}

% Publications from a BibTeX file without multibib
%  for numerical labels: \renewcommand{\bibliographyitemlabel}{\@biblabel{\arabic{enumiv}}}% CONSIDER MERGING WITH PREAMBLE PART
%  to redefine the heading string ("Publications"): \renewcommand{\refname}{Articles}
\nocite{*}
\bibliographystyle{plain}
%\bibliography{publications}                        % 'publications' is the name of a BibTeX file

% Publications from a BibTeX file using the multibib package
%\section{Publications}
%\nocitebook{book1,book2}
%\bibliographystylebook{plain}
%\bibliographybook{publications}                   % 'publications' is the name of a BibTeX file
%\nocitemisc{misc1,misc2,misc3}
%\bibliographystylemisc{plain}
%\bibliographymisc{publications}                   % 'publications' is the name of a BibTeX file

%-----       letter       ---------------------------------------------------------

\end{document}


%% end of file `template.tex'.
